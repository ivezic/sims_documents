\documentclass[12pt,preprint]{aastex}
\usepackage{amsmath}

\begin{document}

\title{Determining Illumination Corrections for (almost) Arbitrary SEDs}

\author{Tim Axelrod}

Suppose we have a set of objects on the sky with SEDs that are
drawn from some parameterized family of SEDs.  Examples of such families
are main sequence stars; DA white dwarfs; and emission line galaxies.
In what follows, we will describe the vector of parameters that
defines a member of the family by $\vec{p}$.  Note that there is no
conceptual difficulty in combining several families into a larger
family, if $\vec{p}$ is suitably defined, and in fact we will want to
do this.  I have argued previously that if we have enough exemplars
from this family in
the set of calibration objects, and a large enough
set of observations of them, the self-calibration process can
determine the illumination correction as a function of $\vec{p}$.  The
illumination corrections determined for SNLS are of this form, with
$\vec{p}$ being just a single color of the object, and the input data
being a relatively small set of rastered star field measurements.

Let's make a precise definition of the illumination correction for an
object with a particular SED, $f(\lambda)$, which is not necessarily a
member of our SED family.  I assume that we have a
broadband photometric flat (``white light'') for band $b$,
$F^{b}_{wl}(\vec{r})$. Note that this is a {\bf photometric} flat,
meaning that each point in the flat is an aperture sum over the {\bf
  image processing} flat, with the aperture centered at that
point. It's largely a matter of taste whether this flat has some
approximate illumination correction applied to it.  If that is done,
and the approximation is a good one, the illumination correction that
we determine from self calibration will be of smaller amplitude.

The illumination correction for a particular object is then defined as

\begin{equation}
IC^{b}_{obj}(\vec{r})~=~\frac{F^{b}_{obj}(\vec{r})}{F^{b}_{wl}(\vec{r})}~=~
\frac{\int {d\lambda \, f_{obj}(\lambda) \, \phi^{b}_{sys}(\vec{r}, \lambda)}}{F^{b}_{wl}(\vec{r})}
\end{equation}

% normalization issue here

Note that if one sets $f_{obj}(\lambda)=1$, one obtains the
illumination correction for the white light flat, which is {\bf not}
unity.  Directly measuring $IC^{b}_{obj}$ is not in general practical, since
it would require a much larger number of observations of the object
than we will have in our survey (perhaps $10^4$ required, $10^2$
available in the survey).  However, if the SED of our object can be
adequately expressed as a linear combination of the SEDs in our
parameterized family, we are in business.  Suppose

\begin{equation}
f_{obj}(\lambda)~=~\int {d\vec{p} \, q(\vec{p}) \, f(\lambda, \vec{p})}
\end{equation}

Then

\begin{align}
IC^{b}_{obj}(\vec{r})~&=~\frac {\int {d\vec{p} \, q(\vec{p}) \, \int
    {d\vec{p} \, q(\vec{p}) \, f(\lambda,
      \vec{p})}}}{F^{b}_{wl}(\vec{r})}\\
    &=~\int {d\vec{p} \, q(\vec{p}) \, IC^{b}(\vec{p}, \vec{r})}
\end{align}

where $IC^{b}(\vec{p}, \vec{r})$ is the illumination correction for
each member of the SED family as determined by self-calibration.

As I see it, the major question revolves around eqn(2). Can we really
mock up a supernova SED, for example, as a linear combination of SEDs of objects
that are
\begin{itemize}
\item abundantly present on the sky
\item not significantly time variable
\end{itemize}

This is an interesting question, which we can investigate right now,
given our library of SEDs.  I do think it's going to be crucial to
include emission line galaxies in our family, as Chris has suggested,
because this will expand the range of representable SEDs quite a lot.

A secondary question is how many representatives of the SED family
near some particular value of the parameters $\vec{p}_0$ are required
for self-calibration to give a good estimate for $IC^{b}(\vec{p}_0,
\vec{r})$.  I suspect we can find a decent way to estimate this, and
determine whether it is practical.

In this scheme, the monochromatic flats are used for two purposes;

\begin{itemize}
\item to transform from calibrated magnitudes to standard magnitudes,
  using only the on-axis values
\item to generate a first guess for the illumination correction
  $IC^{b}(\vec{p}, \vec{r})$, which will speed up (or perhaps even
  make possible) the convergence of self-calibration
\end{itemize}   
 
\end{document}
