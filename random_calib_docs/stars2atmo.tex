\documentclass[12pt,preprint]{aastex}
\usepackage{url}
\usepackage{natbib}
\usepackage{graphicx}
%\usepackage{subfig}
\usepackage{fixltx2e}
\usepackage{hyperref}
\usepackage{boxedminipage}
\usepackage{pdflscape}

%%%%%%%%%%%%%%%%%%%%%%%%%%%%%%%%%%%%%%%%%%%%%%%%%%%%
%%% author-defined commands
\newcommand\x         {\hbox{$\times$}}
\def\mic              {\hbox{$\mu{\rm m}$}}
\def\about            {\hbox{$\sim$}}
\def\Mo               {\hbox{$M_{\odot}$}}
\def\Lo               {\hbox{$L_{\odot}$}}
\newcommand{\water}   {H\textsubscript{2}O}
\newcommand{\ozone}    {O\textsubscript{3}}
\newcommand{\oxy}     {O\textsubscript{2}}

%\captionsetup[figure]{labelformat=simple}
%%%%%%%%%%%%%%%%%%%%%%%%%%%%%%%%%%%%%%%%%%%%%%%%%%%%


\begin{document}

\title{Constraints on Atmospheric Transmissivity from Stellar Photometry}

\author{
Joachim Moeyens\altaffilmark{1}, \\  
{\v Z}eljko Ivezi{\'c}\altaffilmark{1},  R. Lynne Jones\altaffilmark{1}, Peter Yoachim\altaffilmark{1}, \\
Robert Lupton\altaffilmark{2} \\
}
\altaffiltext{1}{University of Washington}
\altaffiltext{2}{Princeton University}


\begin{abstract}
This document will describe the results of analysis by Joachim. 
\end{abstract}

%\tableofcontents

\section{Introduction}

This document will describe the results of analysis by Joachim which aims
to quantify how well we can constrain atmospheric transmissivity using stellar photometry. 



\subsection{Definitions of Relevant Photometric Quantities} 

The following nomenclature is adopted from LSST document LSE-180 (Jones et al.; ``Level 2 Calibration Plan''). 

Given a ``normalized bandpass response function'' for an observation,
\begin{equation}
\label{eqn:PhiDef}
   \phi_b^{obs}(\lambda) =  {
     {S^{atm}(\lambda)\, S_b^{sys}(\lambda) \,
       \lambda^{-1}} \over
     \int_0^\infty { {S^{atm}(\lambda) \,
         S_b^{sys}(\lambda) \, \lambda^{-1}} \,d\lambda}},
\end{equation}
the in-band flux at the top of the atmosphere is defined as
\begin{equation}
\label{eqn:Fb}
F_b^{obs} = \int_0^\infty {F_\nu(\lambda) \,\phi_b^{obs}(\lambda) \, d\lambda}.
\end{equation}
The normalization of $F_b^{obs}$ corresponds to the top of the atmosphere because 
$\phi_b$ only represents {\it shape} information about the bandpass, as by definition
\begin{equation}
\int_0^\infty {\phi_b(\lambda)  d\lambda}=1. 
\end{equation}

In eq.~\ref{eqn:PhiDef}, $S^{atm}(\lambda)$ is the (dimensionless) probability that a photon of 
wavelength $\lambda$ makes it through the atmosphere, and $S_b^{sys}(\lambda)$ is the 
(dimensionless) probability that a photon will pass through the telescope's optical path to be
converted into an ADU count, and includes the mirror reflectivities, lens transmissions, filter
transmissions, and detector sensitivities. Both $S^{atm}$ and $S_b^{sys}$ can vary with position 
within the field of view, and with time. 

The difference in magnitudes between two observations of a temporally constant source obtained with 
different $\phi_b(\lambda)$ is given by 
\begin{equation}
\label{eqn:Delta_m}
\Delta m_b^{obs}  \equiv m_b^{nat} - m_b^{std}  =  -2.5 \, log_{10} \,  \left( { \int_0^\infty {F_\nu(\lambda) \,
    \phi_b^{obs}(\lambda) \, d\lambda} \over \int_0^\infty {F_\nu(\lambda) \,
    \phi_b^{std}(\lambda) \, d\lambda}} \right)
\end{equation}
where $\phi_b^{std}(\lambda)$ corresponds to a ``standard'' bandpass, and $m_b^{nat}$ are
known as ``natural'' magnitudes. Note that $\Delta m_b^{obs}$ depends on the {\it shape} of the 
source spectrum, $F_\nu(\lambda)$ and the {\it shape} of the bandpass
$\phi_b^{obs}(\lambda)$; for a source with flat (constant with respect to wavelength) SED (spectral
energy distribution), $\Delta m_b^{obs}=0$ because the integral of $\phi_b(\lambda)$ is always one by definition. 

Assuming that $\Delta m_b^{obs} \ll 1$, 
\begin{equation}
  \Delta m_b^{obs}   \approx - \, \int_0^\infty { \Delta\phi_b^{obs}(\lambda) \,  {F_\nu(\lambda) \over F_b^{std}} \, d\lambda},
\end{equation}
where $\Delta\phi_b^{obs}(\lambda) = \phi_b^{obs}(\lambda) - \phi_b^{std}(\lambda)$. The integral of 
$\Delta\phi_b^{obs}(\lambda)$ over wavelength is always zero. 

Unless $F_\nu(\lambda)$ is a flat SED ($F_\nu(\lambda)=$constant), $F_b^{obs}$ will vary with 
changes in $\phi_b^{obs}(\lambda)$ due simply to changes in the bandpass shape,
such as changes with position in the focal plane or differing atmospheric absorption 
characteristics, {\it even if the source is non-variable}. This behavior can be used to constrain
$\phi_b^{obs}(\lambda)$ when $\Delta m_b^{obs}$ is known for a large number of sources
with non-flat SEDs. 




\subsection{Atmospheric Extinction} 

Atmospheric extinction can be modeled as due to five components: molecular (Rayleigh) scattering, 
aerosol (Mie) scattering, and molecular absorption by ozone (O$_3$), water vapor and combined
O$_2$/trace species (see Section 5.2 in LSST document LSE-180),
\begin{equation}
\label{eq:atmModel}
 S^{atm}(\lambda) = \Pi_{k=1}^5 \, {\rm e}^{- t_k \, \tau_k^{std}(\lambda, X)} 
                           =  {\rm e}^{-\sum_{k=1}^5  t_k \, \tau_k^{std}(\lambda, X)}. 
\end{equation}
Here $X$ is airmass, $t_k$ is the ratio of the column density (or optical depth) of the $k$-th 
component and its value for the standard atmosphere, and $\tau_k^{std}(\lambda, X)$ is the 
wavelength-dependent optical depth of the $k$-th component for the standard 
atmosphere (with $t_k=1$) viewed at airmass $X$. For the two scattering components and ozone, the {\it shape} of 
$\tau_k^{std}$ (variation with wavelength) is practically independent of airmass (and column
density) 
\begin{equation}
         \tau_{scat}^{std}(\lambda, X) = X \, \tau_{scat}^{std}(\lambda, X=1),
\end{equation}
and analogously for O$_3$. For molecular absorption by H$_2$O and O$_2$, even the 
shape of $\tau_k^{std}$ varies with airmass (we don't know yet whether these shapes
vary when $X$ is fixed and $t_k$ are varied - work in progress...). 

The MODTRAN code can be used to tabulate $\tau_k^{std}(\lambda, X)$ for all components. 
For example, file {\it Pachon\_MODTRAN.10.7sc}  (from directory calib\_doc/level2/code)
lists transmission functions (i.e. $e^{-\tau_k^{std}(\lambda, X=1)}$) for Rayleigh scattering  and 
molecular absorption by H$_2$O, O$_2$, and O$_3$ (columns 9, 3, 4 and 5). Analogous 
files for other values of airmass (up to $X=2.5$) can be found in the same directory.

To allow for a general variation of the aerosol scattering (e.g. due to varying particle 
size distribution), it can be modeled as 
\begin{equation}
      \tau_{aerosol}(\lambda, X, \alpha) = 0.1 \, X \, \left({550 \, {\rm nm} \over \lambda} \right)^\alpha
\end{equation}
with $\alpha=1.7$ for the standard atmosphere and a plausible range $0.5 < \alpha < 2.5$. 

Thus, for given airmass $X$ and wavelength, the atmospheric extinction is fully determined
by six free parameters, $\alpha$ and $t_k$ ($k=1..5$), which measure deviations from the
standard atmospheric extinction curve. 



\subsection{Statement of the Problem} 

Assume that both $S^{sys}_b(\lambda)$ and $S^{atm,std}(\lambda)$ are known (which
implies that $\phi_b^{std}(\lambda)$ is also known). Choose the values of the six free 
atmospheric parameters ($\alpha$ and $t_k$) and generate $\Delta m_b^{obs}$
(using $t_k=1$ for all components and $\alpha=1.7$ generates standard magnitudes). 

Given $\Delta m_b^{obs}$ for a large number of sources with non-flat SEDs
and their known standard colors ($u-g$, $g-r$, $r-i$, $i-z$, $z-y$), how well can we
constrain $\phi_b^{obs}(\lambda)$, and in turn $S^{atm,obs}(\lambda)$ and the corresponding 
six free atmospheric parameters?  As the performance metric, we will eventually use the 
root-mean-square scatter between estimated and true $\Delta m_b^{obs}$ evaluated for Type Ia SNe. 



\section{Analysis} 

Assume that airmass $X=2.0$ and that $\alpha=0.5, 2.5$, and each $t_k=0.5, 2.0$.
We are comparing these $2^6 = 64$ modified atmospheres to the standard atmosphere
with $X=1$, $\alpha=1.7$ and $t_k=1$. To initially simplify the problem, let's only
consider the six cases with $\alpha=2$ and $t_k=2.0$ set one by one, and the 
seventh case of standard atmosphere and $X=2$. 


\subsection{Initial Analysis} 

For each of the seven atmospheres: 

\begin{enumerate} 
\item Plot $\Delta \phi_b(\lambda)$ for all atmospheres (7 plots such as the
bottom panel in fig. 6 from I07). 
\item
How large are $\Delta m_b^{obs}$? 
For each band $b$ ($b=ugrizy$), plot $\Delta m_b^{obs}$ vs. (standard) $g-i$ color
(such as fig. 32 from LSE-180). 
% appropriate standard color ($u-g$ for $u$, $g-r$ for $g$, etc. and $z-y$ for $z$ and $y$). 
% FOR LATER: 
%\item
%{\bf Does traditional calibration method work?} 
%For a given atmosphere and a range of airmass $X$, find the best-fit
%parameters using model 
%$\Delta m_b^{obs} = \alpha + \beta X + \gamma C + \delta X C$
%($\alpha, \beta, \gamma, \delta$ are free parameters and $C$ is the color 
%corresponding to band $b$, as above). What is the root-mean-square 
%scatter? Plot residuals vs. $X$ and vs. $C$ (perhaps color coded in a
%single $X$ vs. $C$ diagram). 
\end{enumerate} 

Lynne's earlier analysis (see e.g. fig. 32 from LSE-180) showed that variations of the 
two aerosol scattering parameters, and the variation of O$_3$ column density, can all 
produce a linear dependence of $\Delta m_g^{obs}$ vs. $g-i$ color (for Kurucz models). 
An interesting question is how much difference there is in corresponding $\Delta \phi_g(\lambda)$
that produce identical $\Delta m_g^{obs}$ vs. $g-i$ curves. 





\subsection{Constraints on $S^{atm}$} 

For a given atmosphere (that is, for fixed values of $X$, $\alpha$ and $t_k$ from eq.~\ref{eq:atmModel}, 
and assuming that $\tau_k^{std}(\lambda, X)$ are known, use $\Delta m_b^{obs}$ and known standard 
colors to constrain $\alpha$ and $t_k$.  

The model for $\Delta m_b^{obs}$ is 
\begin{equation}
  \Delta m_b^{obs}  = \int_0^\infty { \left( \phi_b^{std}(\lambda) - \phi_b^{obs}(\lambda|{\bf p}) \right) \,  {F_\nu(\lambda) \over F_b^{std}} \, d\lambda},
\end{equation}
where the vector of model parameters ${\bf p} = (\alpha; t_k, k=1..5)$. 

Using best-fit $t_k$, compute $\Delta m_b^{best-fit}$ and compare to 
$\Delta m_b^{obs}$, for both Kurucz models and Type Ia SNe. What is the root-mean-square 
scatter? Plot residuals vs. color. 

\end{document}
